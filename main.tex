\documentclass[sigconf,natbib=false]{acmart}
\usepackage{listings}
\usepackage{hyperref}
\usepackage{csquotes}
\usepackage[all]{foreign}

\usepackage[inline]{enumitem}
\newlist{paragraphs}{enumerate*}{1}
\setlist[paragraphs]{
    label=(\theparagraphsi),
    itemjoin=\newline\hspace*{\parindent}
}

\usepackage[capitalize]{cleveref}
\newlist{courses}{enumerate}{1}
\setlist[courses]{label={(C\arabic*)},ref={C\arabic*}}
\crefname{coursesi}{course}{courses}
\Crefname{coursesi}{Course}{Courses}
\crefformat{footnote}{#2\footnotemark[#1]#3}

\usepackage{biblatex}
\addbibresource{main.bib}
\usepackage{acro}
\DeclareAcronym{POT}{%
  short = POT,
  long = peer observation of teaching,
  cite = PeerObservation
}



%% Rights management information.  This information is sent to you
%% when you complete the rights form.  These commands have SAMPLE
%% values in them; it is your responsibility as an author to replace
%% the commands and values with those provided to you when you
%% complete the rights form.
\setcopyright{acmcopyright}
\copyrightyear{2021}
\acmYear{2021}
% \acmDOI{10.1145/1122445.1122456}

%% These commands are for a PROCEEDINGS abstract or paper.
\acmConference[ITiCSE 2021] {26th ACM Conference on Innovation and Technology 
  in Computer Science Education V. 2}{June 26--July 1, 2021}{Virtual Event, 
Germany}
\acmBooktitle{26th ACM Conference on Innovation and Technology in Computer 
  Science Education V. 2 (ITiCSE 2021), June 26--July 1, 2021, Virtual Event, 
Germany}
\acmPrice{15.00}
\acmISBN{978-1-4503-8397-4/21/06}
\acmDOI{10.1145/XXXXXX.XXXXXX}
% Authors, replace the red X's with your assigned DOI string during the 
% rightsreview eform process.

%%
%% Submission ID.
%% Use this when submitting an article to a sponsored event. You'll
%% receive a unique submission ID from the organizers
%% of the event, and this ID should be used as the parameter to this command.
%%\acmSubmissionID{123-A56-BU3}

\settopmatter{printacmref=true}
\begin{document}

%%
%% The "title" command has an optional parameter,
%% allowing the author to define a "short title" to be used in page headers.
\title{When Flying Blind, Bring a Co-pilot}
\subtitle{Informal Peer Observation and Cooperative Teaching During Remote Teaching}

%%
%% The "author" command and its associated commands are used to define
%% the authors and their affiliations.
%% Of note is the shared affiliation of the first two authors, and the
%% "authornote" and "authornotemark" commands
%% used to denote shared contribution to the research.

\author{Daniel Bosk}
\orcid{}
\affiliation{%
  \institution{KTH Royal Institute of Technology}
  \city{Stockholm}
  \country{Sweden}
}
\email{dbosk@kth.se}

\author{Richard Glassey}
\orcid{}
\affiliation{%
  \institution{KTH Royal Institute of Technology}
  \city{Stockholm}
  \country{Sweden}
}
\email{glassey@kth.se}
\fancyhead{}

%%
%% By default, the full list of authors will be used in the page
%% headers. Often, this list is too long, and will overlap
%% other information printed in the page headers. This command allows
%% the author to define a more concise list
%% of authors' names for this purpose.
% \renewcommand{\shortauthors}{Bosk and Glassey}

%%
%% The abstract is a short summary of the work to be presented in the
%% article.
\begin{abstract}

Emergency remote teaching has come as a shock to many teachers and students 
alike. The shift to meeting students online has been disorienting, making 
traditional forms of interaction difficult or impossible to replicate. In 
response, we suggest that teachers become co-pilots for each other --- joining 
lectures and extending the abilities of a solo teacher. By doing so, there are 
clear and distinct benefits for students, the teacher, and the co-pilot, with 
almost no barrier to entry and very little preparation required. Whilst there 
is a time cost that is mostly unavoidable, we feel this is well spent and acts 
as a gateway to more established pedagogical practices, such as peer 
observation and cooperative teaching.

\end{abstract}

%%
%% The code below is generated by the tool at http://dl.acm.org/ccs.cfm.
%% Please copy and paste the code instead of the example below.
%%
\begin{CCSXML}

\end{CCSXML}

% \ccsdesc[500]{Computer systems organization~Embedded systems}
% \ccsdesc[300]{Computer systems organization~Redundancy}
% \ccsdesc{Computer systems organization~Robotics}
% \ccsdesc[100]{Networks~Network reliability}

%%
%% Keywords. The author(s) should pick words that accurately describe
%% the work being presented. Separate the keywords with commas.
\keywords{online teaching, co-teaching, collaborative teaching, cooperative 
teaching}

%%
%% This command processes the author and affiliation and title
%% information and builds the first part of the formatted document.
\maketitle

\section{Introduction}

The skill set for remote teaching is one that many teachers have needed to  build throughout the Covid-19 pandemic, but with little time for careful planning or reflection on practice. Besides the technological challenges, teachers have also encountered difficulties in perceiving the teaching space; no longer can we read the room with a quick glance, rather we gaze upon grids of black squares. Efforts to \enquote{read the Zoom}, such as asking students to turn their camera on might work in small classes, but for more than 30 students, services like Zoom create several pages of video thumbnails to scroll through. Furthermore, while sharing the screen (\eg presenting slides), the teacher can at best only see a few students.

The purpose of this contribution is to share our positive experiences of remote teaching, with a focus on strategies on improving remote lectures\footnote{And perhaps lectures in general.}. We experimented with a \emph{co-pilot approach} to remote lectures over Zoom, where one academic takes the lead for presenting the lecture content, whilst the second academic, the co-pilot, takes care of various duties like managing the chat area, keeping time, reminders, confirmations, and engaging in discussion before, during and at the conclusion of the lecture. The end result was a more enjoyable experience for the students and a less lonely experience for the teacher. Furthermore, this turns out to be a gateway to experiencing aspects of peer observation and cooperative teaching, but without the need for preparation.

\section{Co-pilot Context}
We had the opportunity to co-pilot in several different courses, with varying amounts of students, and different configurations of teachers and roles. Here, we briefly describe the courses:

\begin{courses}
  \item\label{datintro20} \textbf{DD1301 Computer Introduction}, a two-week 
    course with a cohort of 265 first year students that is shared across 
    several degree programmes. It consisted of three one-hour lectures that 
    were given jointly by the authors\footnote{\label{first-together}First time 
    giving this course together.}.

  \item\label{inda20} \textbf{DD1337 Programming}, a 10-week course given to a 
    cohort of 200 first-year computer science students. It consisted of 10 
    two-hour lectures that were given jointly by the 
    authors\cref{first-together}.

  \item\label{prgi20} \textbf{DD1315 Programming Techniques and Matlab}, a 
    10-week course with a cohort of 163 first year industrial economy students. 
    It consisted of 13 two-hour lectures that were given by one of the authors 
    and a teaching assistant\footnote{First time working together.}.
\end{courses}

In terms of planning, informal discussions occurred between the authors before the academic term. We agreed that it would be sensible to sit-in on each other's lectures to get a sense of how it was going. Both had a mixed experience of remote teaching previously, however this was the first time all of the  lectures would be remote, using Zoom. We opted to expect the unexpected.

\ref{datintro20} was our initial experiment. The students generated lots of 
chat, \eg one of the lectures generated $265$ messages in one hour. After this, 
we realized that it would be difficult to be a single teacher managing the chat 
and giving the lecture at the same time, and the concept of a \emph{proactive 
co-pilot} was quickly hatched. \ref{inda20} was also given by the authors, but 
over a longer time-frame. Each weekly lecture gave us the opportunity to 
experiment with the co-piloting and learn where it was most helpful. The 
students were active in chat during all of the lectures and they produced more 
messages than was possible for a teacher to handle, see \cref{PrgiMessageDist} 
for details. As the courses concluded, we took the time to reflect on our 
experiences, as well as poll the students via course evaluations for their 
opinions about remote teaching.

\begin{table}
    \centering
    \begin{tabular}{lr lr lr}
        \toprule
        \multicolumn{2}{c}{Course \ref{prgi20} (2 hours)} &
        \multicolumn{2}{c}{Course \ref{inda20} (2 hours)} &
        \multicolumn{2}{c}{Course \ref{datintro20} (1 hour)} \\
        \midrule
        Lecture 1 & 176 & Lecture 1 & 486 & Lecture 1 & 265 \\
        Lecture 2 & 215 & Lecture 2 & 576 & Lecture 2 & 174 \\
        Lecture 3 & 102 & Lecture 3 & 512 & Lecture 3 & 161 \\
        Lecture 4 & 60 & Lecture 4 & 349 \\
        Lecture 5 & 117 & Lecture 5 & 250 \\
        Lecture 6 & 144 & Lecture 6 & 98 \\
        Lecture 7 & 81 & Lecture 7 & 144 \\
        Lecture 8 & 150 & Lecture 8 & 207 \\
        Lecture 9 & 119 & Lecture 9 & 196 \\
        Lecture 10 & 113 & Lecture 10 & 193 \\
        Lecture 11 & 117 \\
        Lecture 12 & 93 \\
        Lecture 13 & 74 \\
        \bottomrule
    \end{tabular}
    \caption{Total number of messages per lecture for courses \ref{prgi20} (163 students), \ref{inda20} (200 students) and \ref{datintro20} (265 students).}
    \label{PrgiMessageDist}
\end{table}

\section{Co-pilot Experience}

Our biggest concern with this abrupt and unplanned change to our courses was the student experience. The main benefits from co-piloting specifically for students included:

\begin{itemize}
  \item There is a better energy in the discussion between teacher and co-pilot, allowing multiple perspectives and expertise on any given topic, as well as informal chit chat at the beginning, middle and end of lecture that lightens the mood.

  \item Students can ask questions in chat and be answered directly by the co-pilot, or the co-pilot can decide to interrupt and share the question with the teacher if it demands attention.
\end{itemize}


\noindent
Students openly complimented the approach in the course evaluation, as the following quotes highlight:

\begin{quote}
The lectures have been very good! Super good that Daniel can keep track of the chat while Ric continues with the lecture, in this way there will be good flow in the lecture at the same time as questions are answered.
\end{quote}

\begin{quote}
Being several who give the lectures is absolutely fantastic! And when there is some part that is \enquote{boring} as a plus \textins{a student who already knows the topic}, there is almost always something interesting in discussing with the chat about.
\end{quote}

\noindent
Beyond the students, there are benefits specifically for the teacher:

\begin{itemize}
  \item You are not alone. Simple audio/visual checks can be made. Reminders on time keeping. Alerts to important discussions happening in the chat area. The co-pilot can also ask good questions pre-emptively, based on the co-pilot's experience and expertise. Another source of feedback on your teaching practice from a trusted and informed colleague.

  \item There is no technical cost. Adding another teacher to the room is easy and feels natural with remote teaching services.

  \item You can focus. The stress of managing all of the new aspects of teaching remotely can make us perform less well than in our regular theatre. Offloading some of this worry really helps to refocus on the content, interaction and learning.
\end{itemize}

\noindent
And finally, the specific benefits for the co-pilot:

\begin{itemize}
  \item The co-pilot sees another way of teaching a topic that they themselves might be teaching too. This makes them reflect on someone else's teaching and their own. In the case of a TA co-pilot, they also gained new insights into the topic.
\end{itemize}

\section{Discussion}

As indicated in the previous section, inviting a co-pilot into your remote teaching was beneficial to all concerned. However, on reflection, much of what we experienced is not particularly novel, when considering the well established pedagogical practices of peer observation of teaching (POT)~\cite{PeerObservation} and cooperative teaching~\cite{bauwens1995cooperative} (or team teaching or co-teaching). What is different with co-piloting is the very low barrier of entry, convenience afforded by remote teaching, and quite honestly, not having the time to properly prepare in the first place for the transition to remote teaching.

At the most basic level is frictionless setup. The teacher and the co-pilot do not have to be physically in the same space. With time constrained teachers, cutting on \enquote{commuting time} across campus decreases the friction and lowers the bar for adoption. The teacher can just send a link, the co-pilot can join from anywhere. In normal times, inviting a colleague to a lecture was unheard of and POT only occurred as a prerequisite of faculty training and promotion.

Despite the simplicity of the concept, teachers have the chance to experience parts of POT and cooperative teaching, without having to spend time and effort. It may very well act as a gateway to embracing these practices, once the ice has been broken by sharing your classroom with another colleague. Flipping the discomfort of being observed in your teaching into a means of improving your practice and reducing your own discomfort with remote teaching appears to be a win-win situation. Also, the ability to effortlessly record teaching sessions is another benefit to both teacher and co-pilot. In traditional POT, the observer acts passively as the recorder and creates their subjective account of the session. With the recording available, both teacher and co-pilot have an objective record of what happened. The natural extension is to add more rigorous analysis, reflection and iteration, as in lesson or learning studies~\cite{NecessaryConditionsOfLearning}.

Perhaps the largest downside of this approach is the \textbf{time cost}. Having two teachers per lecture takes time that is already in short supply. This particular point has already been raised in the context of cooperative teaching and as such we do not have much to add. Our main recommendation would be to take the potential described here and have an experienced teaching assistant fulfil the role, or alternatively have a rotating schedule with your colleagues that share a passing interest in your course.

\section*{Conclusion}

Remote teaching is new and ongoing for many. Some adapt with ease, but others find it uncomfortable and disorienting. One of the simplest ways to recover your sense of what is going on in your class is to invite a co-pilot to your lectures. This will improve the experience for your students, yourself and your co-pilot.

\printbibliography

\end{document}
\endinput
