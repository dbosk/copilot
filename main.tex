\documentclass[sigconf,natbib=false]{acmart}
\usepackage{listings}
\usepackage{hyperref}
\usepackage{csquotes}
\usepackage[all]{foreign}

\usepackage[inline]{enumitem}
\newlist{paragraphs}{enumerate*}{1}
\setlist[paragraphs]{
    label=(\theparagraphsi),
    itemjoin=\newline\hspace*{\parindent}
}

\usepackage[capitalize]{cleveref}
\newlist{courses}{enumerate}{1}
\setlist[courses]{label={(C\arabic*)},ref={C\arabic*}}
\crefname{coursesi}{course}{courses}
\Crefname{coursesi}{Course}{Courses}
\crefformat{footnote}{#2\footnotemark[#1]#3}

\usepackage{biblatex}
\addbibresource{main.bib}
\usepackage{acro}
\DeclareAcronym{POT}{%
  short = POT,
  long = peer observation of teaching,
}



%% Rights management information.  This information is sent to you
%% when you complete the rights form.  These commands have SAMPLE
%% values in them; it is your responsibility as an author to replace
%% the commands and values with those provided to you when you
%% complete the rights form.
\setcopyright{acmcopyright}
\copyrightyear{2018}
\acmYear{2018}
\acmDOI{10.1145/1122445.1122456}

%% These commands are for a PROCEEDINGS abstract or paper.
\acmConference[Woodstock '18]{Woodstock '18: ACM Symposium on Neural
  Gaze Detection}{June 03--05, 2018}{Woodstock, NY}
\acmBooktitle{Woodstock '18: ACM Symposium on Neural Gaze Detection,
  June 03--05, 2018, Woodstock, NY}
\acmPrice{15.00}
\acmISBN{978-1-4503-XXXX-X/18/06}

%%
%% Submission ID.
%% Use this when submitting an article to a sponsored event. You'll
%% receive a unique submission ID from the organizers
%% of the event, and this ID should be used as the parameter to this command.
%%\acmSubmissionID{123-A56-BU3}

\begin{document}

%%
%% The "title" command has an optional parameter,
%% allowing the author to define a "short title" to be used in page headers.
\title{When flying blind, bring a co-pilot}
\subtitle{Informal Peer Observation and Cooperation During Remote Teaching}

%%
%% The "author" command and its associated commands are used to define
%% the authors and their affiliations.
%% Of note is the shared affiliation of the first two authors, and the
%% "authornote" and "authornotemark" commands
%% used to denote shared contribution to the research.

\author{Daniel Bosk}
\orcid{}
\affiliation{%
  \institution{KTH Royal Institute of Technology}
  \city{Stockholm}
  \country{Sweden}
}
\email{dbosk@kth.se}

\author{Richard Glassey}
\orcid{}
\affiliation{%
  \institution{KTH Royal Institute of Technology}
  \city{Stockholm}
  \country{Sweden}
}
\email{glassey@kth.se}

%%
%% By default, the full list of authors will be used in the page
%% headers. Often, this list is too long, and will overlap
%% other information printed in the page headers. This command allows
%% the author to define a more concise list
%% of authors' names for this purpose.
% \renewcommand{\shortauthors}{Bosk and Glassey}

%%
%% The abstract is a short summary of the work to be presented in the
%% article.
\begin{abstract}
Todo :)

\begin{itemize}
    \item Expand sections with more content
    \item Consider adding more quantitative data
    \item Add Team / Cooperative teaching to discussion
\end{itemize}

\end{abstract}



%%
%% The code below is generated by the tool at http://dl.acm.org/ccs.cfm.
%% Please copy and paste the code instead of the example below.
%%
\begin{CCSXML}

\end{CCSXML}

% \ccsdesc[500]{Computer systems organization~Embedded systems}
% \ccsdesc[300]{Computer systems organization~Redundancy}
% \ccsdesc{Computer systems organization~Robotics}
% \ccsdesc[100]{Networks~Network reliability}

%%
%% Keywords. The author(s) should pick words that accurately describe
%% the work being presented. Separate the keywords with commas.
\keywords{}

%%
%% This command processes the author and affiliation and title
%% information and builds the first part of the formatted document.
\maketitle

\section{Introduction}

The skill set surrounding remote teaching is one that many of us are currently 
building. Besides the new technological challenges to our normal practice, we 
also encounter challenges in our experience of teaching; no longer can we read 
the room, rather we can only gaze out into a grid of mostly black squares that 
Zoom presents to us.
Efforts to \enquote{read the Zoom}, such as asking students to turn camera on 
might work in small classes, but for classes of more than 30 students, Zoom 
creates several pages of video feeds to scroll through.
And while sharing the screen (\eg slides), the teacher can only see very few 
students at the same time.

The purpose of this contribution is to share our positive experiences of remote 
teaching, with a focus on strategies on improving remote lectures\footnote{%
  And perhaps lectures in general.
}.
We experimented with a co-pilot approach to remote lectures over Zoom, where 
one academic takes the lead for presenting the lecture content, whilst the 
second academic, the co-pilot, takes care of various duties like managing the 
chat area, keeping time, reminders, confirmations, and engaging in discussion 
before, during and at the conclusion of the lecture.
The end result was a more enjoyable experience for the students and a less 
lonely experience for the academic.
Furthermore, this turns out to be a great way to naturally adopt the 
pedagogical practice of peer observation of teaching.


\section{Co-pilot Origins}

So far, we have delivered the following courses in this way:
\begin{courses}
  \item\label{datintro20} DD1301 Computer introduction, a two-week course with 
    a cohort of 265 first year students distributed over several programmes.
    It consisted of three one-hour lectures.
    Given jointly by the authors.

  \item\label{inda20} DD1337 Programming, a 10-week course to with a cohort of 
    200 first-year computer science (CDATE) students.
    It consisted of 13 two-hour lectures.
    Given jointly by the authors.

  \item\label{prgi20} DD1315 Programming Techniques and Matlab, a 10-week 
    course with a cohort of 163 first year CINEK students.
    It consisted of 16 two-hour lectures, two per week during the active weeks.
    Given by one of the authors and a teaching assistant.
\end{courses}

\ref{datintro20} was our initial experiment.
The students generated quite some traffic in the chat, \eg third lecture generated
344
messages\footnote{%
  Public and private for both teachers.
} during a one-hour lecture.
After this experience we realized that it would be difficult to be only one 
person managing the chat while giving the lecture at the same time.

\ref{inda20} was also given by the authors.
Each weekly lecture has given us the opportunity to experiment with the 
co-piloting approach and learn where it is helpful.
As the course concludes it seems opportune to share these findings with other 
teachers at KTH.


\section{Student, Teacher (and Co-pilot) Experience}

\paragraph{For students}

The benefits for the students are in terms of qualities of the teaching.

\begin{itemize}
  \item There is a better energy in the discussion between teacher and 
    co-pilot, allowing multiple perspectives on any given topic, as well as 
    informal chit chat at the beginning, middle and end of lecture.

  \item Students can ask questions in chat and have them answered directly by 
    the co-pilot, or the co-pilot can decide to interrupt and share the 
    question with the teacher.
\end{itemize}

Some quotes from the students:
\begin{quote}
The lectures have been very good! Super good that Daniel can keep track of
the chat while Ric continues with the lecture, in this way there will be
good flow in the lecture at the same time as questions are answered.
\end{quote}

\begin{quote}
Being several who give the lectures is absolutely fantastic! And when there
is some part that is \enquote{boring} as a plus \textins{a student who already 
knows the topic}, there is almost always something
interesting in discussing with the chat about.
\end{quote}


\paragraph{For teachers and co-pilots}

Benefits for the teachers:

\begin{itemize}
  \item You are not alone. Simple audio/visual checks can be made. Reminders on 
    time keeping. Alerts to important discussions happening in the chat area.
    The co-pilot can also ask good questions pre-emptively, based on the co-pilots experience.
    Feedback on your teaching from a colleague.

  \item There is no technical cost. Simply adding another teacher to the room 
    is easy and feels natural.

  \item You can focus. The stress of managing all of the new aspects of 
    teaching remotely can make us perform less well than in our regular 
    theatre. Offloading some of this worry really helps to refocus on the 
    content and learning.
\end{itemize}

Benefits for the co-pilots:

\begin{itemize}
  \item The peer-co-pilot see another way of teaching a topic that themself 
    might be teaching too.
    This makes them reflect on someone else's teaching and their own.
    In the case of a TA co-pilot, they also gained some new insights into the
    topic --- despite being experienced.
\end{itemize}

\section{Discussion}

Whilst not the intention, co-piloting turned out to be a great way of naturally 
introducing a variant of 
\ac{POT}~\cite[\cf][]{PeerObservation,ReflectivePeerObservation},
a critical 
pedagogical practice for helping to develop and maintain awareness of how you 
perform in class from someone else's point of view.

One benefit of doing this for online teaching is the frictionless setup. The 
teacher and the co-pilot do not have to be physically in the same space.
With time constrained teachers, cutting on \enquote{commuting time} across 
campus decreases the friction and lowers the bar for adoption.
The teacher can just send a link, the co-pilot can join from anywhere.

Another benefit of doing this for online teaching is the ease of recording the 
lecture.
Traditionally, the peer serves passively as the 
\enquote{recorder}~\cite{PeerObservation,ReflectivePeerObservation}.
In the online setting, the peer can actively participate and there will still 
be a record of observations to analyse afterwards.

Perhaps the largest downside of this approach is the \textbf{time cost}. Having 
two teachers per lecture runs the risk of stealing too much time. Our 
recommendation would be to take the potential described here and have an 
experienced teaching assistant fulfil the role\footnote{%
  That worked very well in our experience.
}, or alternatively have a partial schedule
with your colleagues.

The natural extension is to add more rigorous analysis, reflection and 
iteration, as in lesson or learning 
studies~\cite{NecessaryConditionsOfLearning}.

\begin{itemize}
    \item 
\end{itemize}

\section*{Conclusion}

Remote teaching is new for many.
One of the simplest ways to recover your sense of what is going on in your class is to invite a co-pilot to your lecture series.
This will also allow you to improve the experience for all.


\printbibliography

\end{document}
\endinput
